\begin{NJUSTAbstractZH}	
	
\qquad 本模板是以程杰、方青云二位学长已完成的南理工\LaTeX 模板为基础,根据教务处本科生毕设论文要求进行修改、封装而成的。正常使用请参照本文即可,如需对模板进行修改请查看style目录下的NJUSTBachelorThesis.cls文件,绝大部分格式内容已封装进此文件内。	

\qquad 本模板已在 Windows 10 / Ubuntu 18.10下通过测试。如在使用过程中有任何问题可以在Github上留言,或是通过\url{kb1000fx@gmail.com}与我联系。

\qquad 欢迎广大南理工学子不断进行修改、完善。

\vspace{8ex}

%% 添加中文关键词,最多5个
\NJUSTKeyWordsZH {南京理工大学}{模板}{本科}

\end{NJUSTAbstractZH}

\newpage

\begin{NJUSTAbstractEN}	
	
%%英文摘要中的题目,两个括号对应题目中的两行
\NJUSTENAbsTitle{Gettysburg Address}{Abraham Lincoln }

\fontspec{SimSun} {\zihao{-4}{
	Four score and seven years ago our fathers brought forth on this continent, a new nation, conceived in Liberty, and dedicated to the proposition that all men are created equal.
	
	Now we are engaged in a great civil war, testing whether that nation, or any nation so conceived and so dedicated, can long endure. We are met on a great battle-field of that war. We have come to dedicate a portion of that field, as a final resting place for those who here gave their lives that nation might live. It is altogether fitting and proper that we should do this.
	
	But, in a larger sense, we can not dedicate -- we can not consecrate -- we can not hallow -- this ground. The brave men, living and dead, who struggled here, have consecrated it, far above our poor power to add or detract. The world will little note, nor long remember what we say here, but it can never forget what they did here. It is for us the living, rather, to be dedicated here to the unfinished work which they who fought here have thus far so nobly advanced. It is rather for us to be here dedicated to the great task remaining before us -- that from these honored dead we take increased devotion to that cause for which they gave the last full measure of devotion -- that we here highly resolve that these dead shall not have died in vain -- that this nation, under God, shall have a new birth of freedom -- and that government of the people, by the people, for the people, shall not perish from the earth.
	
	\vspace{8ex}
}}
%% 添加英文keyword,最多可添加5个	
\NJUSTKeyWordsEN {NJSUT}{Nanjing}{bachelor}{thesis}

\end{NJUSTAbstractEN}	


